\chapter{Conclusions} \label{conclusion}

\epigraph{2in}{People fear death even more than pain. It's strange that they fear death. Life hurts a lot more than death. At the point of death, the pain is over. Yeah, I guess it is a friend.}{Jim Morrison}{}



Protecting operating system kernels from malware that executes with the same privilege level is an extremely challenging task for which, at the moment, there is no winner either from the community of security researchers nor attackers. This fact is mainly due to the nature of the shared environment in which both trusted and malicious code operate.
A mitigation to attacks with kernel level malware can be achieved when trusted code has been isolated and any external attempt to tamper with it will fail. 

Virtualisation technology offers the aforementioned required isolation capabilities, but at the cost of an additional layer referred to as \emph{hypervisor}, on top of which regular operating systems can be executed. Although we are aware of attacks to the hypervisor that can compromise the entire virtualisation infrastructure \cite{hyperattack, Desnos:2011:DHR:1938158.1938205,Embleton:2008:SRN:1460877.1460892, Gebhardt:2008:HPH:2303959.2304226}, we believe that such attacks can occur under assumptions that are stronger than the ones explained in this work, such as physical access or faulty hardware. We demonstrated how the isolation between a hypervisor and a guest operating system can be used to build a non-bypassable protection system against kernel-level malware. Despite some limitations of the described protection system, regarding the type of kernel code that can be protected, the overall attack surface of the target system results dramatically reduced, giving the attacker very few chances to launch malicious operations. 
In general, isolation is a common requirement of all protection strategies dealing with kernel security. 
In traditional systems in which the countermeasure and the code to be protected share the same space and privileges, it is highly unlikely that trusted code will still be executed even after the kernel has been compromised. The claims made by researchers or secure software vendors about the effectiveness of these types of security measures can be achieved if and only if the isolation requirement is fulfilled.
 
We have contributed to this end by designing a framework that sets a protected environment within the target system and enforces the execution of trusted code from the hypervisor. The aforementioned enforcement of trusted code results decoupled from the target system. Therefore, there are no viable ways for an attacker to tamper with the countermeasure in order to postpone security checkings or to circumvent them completely. In our opinion, two important features, both present in the proposed framework, make it suitable for production systems: isolation and performance.

Isolation allows the execution of the trusted code even after the operating system has been compromised. Any countermeasure in place within a compromised kernel cannot be guaranteed to provide the functionality it has been designed for. This was not the case in our framework.  

Moreover, by setting most of the secure environment within the target system, our framework can operate with an almost native performance impact. 
The countermeasures related to kernel attacks we propose in this work have been designed with performance and size of instrumentation code in mind. Despite the improvements in hardware supported virtualisation technology, we believe that another fundamental property that can impede a security measure's chances of competing in the marketplace is the complexity of integrating it into existing solutions.
The hybrid approach - extensively used in the countermeasures presented in this work - of extending the target kernel with a trusted module that bridges communication to the hypervisor is revealed to be an effective strategy that can be considered even for those systems that cannot be modified for intellectual property reasons or because their source code is not available. Once the trusted module is no longer needed, it is unloaded from the target kernel in order to remove it from the attack surface and to reduce the chances of circumvention even further.
Despite the recurrent mechanism provided by virtualisation technology by which the execution of the guest can be arbitrated by the hypervisor, finding a common pattern to the plethora of attacks that might compromise virtualised operating systems is challenging.

Another area we focus on belongs to the field of web browser security, a topic discussed in the second part of the thesis.
Owing to the recent transformation of the web browser into one of the most important elements in today's computer 
usage and a recent \emph{heap-based} attack that can circumvent some of the most effective countermeasures, we developed a lightweight security measure that not only revealed to be effective but it is also affected by negligible performance impact. We focused on \emph{heap-spraying} attacks, by which objects of malicious content can be allocated on the browser's heap via script languages such as Javascript. An essential requirement to deploy heap-spraying attacks successfully is homogeneity of data. When the malicious object has been allocated in the form of a homogeneous array, it becomes relatively easy for an attacker to forward the instruction pointer of the running system to a location within the aforementioned array and start execution of its content.
By introducing diversity via random interruptions of special bytes we can successfully prevent the execution of malicious code stored on the heap. 

Another challenging aspect when shifting to a newer technology involves designing countermeasures that can take 
advantage of the new features and can be easily integrated with the purpose of improving both performance and security. We contributed to this end by providing a strategy to integrate secure web browsers with hypervisor technology. This might have an impact in those cases in which applications are delivered on demand, as in a \emph{desktop virtualisation} setting.





\section{Future work}
Despite the numerous contributions to tackling kernel malware, we believe that further research is needed in this area. In the virtualisation-based rootkit protection system described in Chapter \ref{hellorootkitty}, we are aware of a consistent limitation that restricts integrity checking to those kernel objects that stay invariant during the system lifetime. 
Although the proposed countermeasure provides an efficient protection against kernel mode rootkits, we believe that there is still room for improvement. We expect that rootkits with higher complexity might target \emph{variable} data structures not only to circumvent a countermeasure like the one we described, but also to achieve a more complex behaviour and inflict further damage. Therefore, we suggest further research that leads to protecting critical kernel objects that are permitted to change during operating system lifetime, such as task structures created at runtime, structures that can be assigned to multiple values, dynamic kernel pointers, etc. as reported in \cite{dynamicdatakernel}. 
The reader is likely to notice that protecting dynamic kernel data is a much more difficult task for which an approach different from the one proposed for static objects must be considered. 

The trusted code enforcement framework proposed in this thesis can be used to perform the difficult task of rootkit prevention in a more dynamic way. This major flexibility is due to the fact that, despite the isolation layer, the trusted code is executing within the target operating system. The number of applications that may take advantage of such a framework is limited by the needs and reader's imagination. For instance, traditional signature-based anti-malware systems are effective only against known rootkits \cite{Mahapatra:2011:OCV:1988997.1989022} and any attempt to protect against rootkits coded with a different style or targeting unprotected areas of the kernel might fail. Moreover, in case of attack, these systems can be deactivated by the malware itself since they are executing within the same system to be protected. Such anti-malware systems can thus benefit from the secure framework we propose in order to operate within the target system and to stay isolated at the same time, making any attempt to be circumvented extremely difficult or not feasible at all. 
 
To conclude, we identify an area that needs the attention of security researchers in light of virtualisation technology, namely \emph{mobile computing}. The trend of mobile devices outselling traditional computers is a consistent evidence of the drift of computing experience in general \cite{mobilemagic, oulasvirta:habits, 6072199}. Moreover, this popularity is stimulating the spread of malware specifically designed for operating systems that execute on  mobile devices \cite{mobilemalware, mobilemalware2, androidmalware, androidmalware2, iosmalware}.
  
We expect that virtualisation technology will affect the mobile arena in the near future. Hardware support to virtualisation might provide an entirely new way of thinking about security for mobile devices, similarly to what has been observed so far.
For instance, the technology for executing corporate and personal identity on top of the same physical mobile device is already present \cite{mobilevirt}. 
The nature of mobile devices, usually equipped with limited hardware resources (such as less computational power, smaller storage capacity, finite  battery life, etc), may place additional constraints on the design of those security countermeasures that will take advantage of mobile virtualisation technology. 


