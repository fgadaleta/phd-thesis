\section*{Abstract}

Computers that are connected to a network have become an integral part of our society and the idea of being continuously connected to the Internet is gaining more acceptance.  Contemporary computer networks are realized using a large number of heterogeneous access technologies, such as Ethernet, DSL, wireless Ethernet, UMTS, etc.   Additionally, network devices have become small enough to be carried around and are used to communicate in public places using publicly available access networks.

Despite the availability of all these access technologies, well-equipped network devices and the use of carefully designed communication software, applications still run into problems when running in such mobile, heterogeneous network environments: network addresses and protocols change, network characteristics (bandwidth, jitter,\dots) fluctuate and network disconnections occur frequently.

These problems lead to four major challenges for the next generation mobility solutions:  First, the mobility solution must support both address and protocol changes. Secondly, when desired, applications must be kept aware of mobility events. Thirdly, switching to another access network must happen in a secure way. Fourthly, it must be possible to deploy the solution in a heterogeneous network where access technologies and communication protocols evolve quickly.

This dissertation contributes a mobility solution architecture that addresses these challenges by introducing a session layer protocol in the protocol stack.  This architecture is realized by two systems:  the Connection Abstractions System (CAS) and the Address Management System (AMS).  The CAS defines a session as a logical communication channel between two applications.  Communication for a CAS session is realized using the transport protocols that are available at that time.  Transport protocol connections that are aborted as a consequence of mobility are replaced by new connections.  If this happens, the CAS maintains communication reliability and optionally informs the application.  The CAS's session protocol can authenticate the moving applications if that is desired.  Protocol changes are enabled by the AMS, which introduces the necessary concepts for developing network application development without prior knowledge of the available communication protocols.  Both systems are implemented and evaluated using the DiPS+ protocol stack framework.% developed in Java.  

