\documentclass[10pt]{book} 

% foefel commando's om citaat en chapter op 1 pagina te krijgen...
\makeatletter
\newcommand\citaat{\if@openright\cleardoublepage\else\clearpage\fi
                     \thispagestyle{plain}
                     }

% chapters that do not start at new page...
\newcommand\chapterNoPageBreak{
                     \thispagestyle{plain}
                     \global\@topnum\z@ 
                     \@afterindentfalse 
                     \secdef\@chapter\@schapter}  
\makeatother

%net
\textheight 18cm %19.3cm	% vele andere: slechts 17.5 
\textwidth 12.5cm %12.3cm  	%lode zelfs 12.5
\oddsidemargin 1.75cm
\evensidemargin 1.75cm

\usepackage[dutch,english]{babel}
\usepackage{epstopdf} 
\usepackage{fancyhdr}
%\usepackage{epigraph}
\usepackage{setspace}
\usepackage{url}
\usepackage{cite}
\usepackage{rotating}
\usepackage{amssymb, amsthm, amsmath, amscd}
\newcommand{\binomial}[2]{\genfrac{(}{)}{0pt}{}{ #1 }{ #2 }}
 
%figures 
\usepackage{graphicx}
\usepackage{subfigure}
\usepackage[babel=false,english=american]{csquotes}

\usepackage[strings]{underscore}

% listings
\usepackage{listings}
\usepackage{courier}
\renewcommand\lstlistlistingname{List of Listings} % default is Listings
\lstset{language=C}

%fancy heading
\usepackage{fancyhdr}
\pagestyle{fancyplain}

% programming code listings
\usepackage{listings}

% boxedverbatim
\usepackage{moreverb}
\usepackage{multirow}

\newcommand\bookepigraph[4]{
\vspace{1em}\hfill{}\begin{minipage}{#1}{\begin{spacing}{0.9}
\small\noindent\textit{#2}\end{spacing}
\vspace{1em}
\hfill{}{#3}\\
 
\vspace{-1em}\begin{flushright}{#4}\end{flushright}}\vspace{2em}
\end{minipage}}


\newcommand\epigraph[3]{
\vspace{1em}\hfill{}\begin{minipage}{#1}{\begin{spacing}{0.9}
\small\noindent\textit{#2}\end{spacing}
\vspace{1em}
\hfill{}{#3}}\vspace{2em}
\end{minipage}}






\addtolength{\headheight}{3pt}         % allow 1 lines in 12pt
\renewcommand{\headrulewidth}{0.4pt}
\renewcommand{\footrulewidth}{0pt}
\renewcommand{\plainheadrulewidth}{0pt}
\renewcommand{\plainfootrulewidth}{0pt}

\renewcommand{\chaptermark}[1]{\markboth{#1}{}}
\renewcommand{\sectionmark}[1]{\markright{\thesection\ #1}}

\lhead[\fancyplain{}{\rm\thepage}]{\fancyplain{}{\rightmark}}
\chead[]{}
\rhead[\fancyplain{}{\leftmark}]{\fancyplain{}{\rm\thepage}}
\lfoot[]{}
\cfoot[]{\fancyplain{\rm\thepage}{}}
\rfoot[]{}

%%%%%%%%%%%%%%%%%%%%%%%%%%%%%%%%%%%%%%%%%%%%%%%%%%%%%%%%%%%%%%%%%%%%%%%%%%%%%%% 
%  Prints your review date and 'Draft Version' (From Josullvn, CS, CMU) 
\newcommand{\reviewtimetoday}[2]{\special{!userdict begin 
/bop-hook{gsave 20 710 translate 45 rotate 0.8 setgray /Times-Roman
findfont 12 scalefont setfont 0 0 moveto (#1) show 0 -12 moveto (#2)
show grestore}def end}}
% You can turn on or off this option. 
%\reviewtimetoday{\today}{Draft Version} 
%%%%%%%%%%%%%%%%%%%%%%%%%%%%%%%%%%%%%%%%%%%%%%%%%%%%%%%%%%%%%%%%%%%%%%%%%%%%%%%

\input xypic
\newcommand{\morfism}[3]{\mbox{$#1: #2 \rightarrow #3$}}
\newcommand{\Set}{{\bf Set}}
\newcommand{\Grf}{{\bf Grf}}
\newcommand{\FinSet}{{\bf FinSet}}
\newcommand{\Dof}{{\bf Dof}}
\newcommand{\FinDof}{{\bf FinDof}}
\newcommand{\Id}{\mbox{\rm Id}}
\newcommand{\Fun}{\mbox{\rm Fun}}
\newcommand{\FinFun}{\mbox{\rm FinFun}}
\newcommand{\Hom}{\mbox{\rm Hom}}
\newcommand{\Nat}{\mbox{\rm Nat}}
\newcommand{\Lex}{\mbox{\rm Lex}}
\newcommand{\Mod}{\mbox{\rm Mod}}
\newcommand{\colim}{\mbox{\rm colim }}
\newcommand{\nat}{\mbox{\rm \bf N}}
\newcommand{\strings}{\mbox{\rm \bf S}}
\newcommand{\timevalues}{\mbox{\rm \bf T}}
\newcommand{\dates}{\mbox{\rm \bf D}}
%\newcommand{\implies}{\Rightarrow}
\newcommand{\equiva}{\Leftrightarrow}
\newcommand{\sigmaalg}{\mbox{$\Sigma \mbox{-\bf Alg}$}}
\newcommand{\pialg}{\mbox{$\Pi \mbox{-\bf Alg}$}}
\newcommand{\Def}{\mbox{\rm Def}}

\newcommand{\et}[1]{\mbox{\scriptsize #1}}

\newcommand{\infer}[3]{$\begin{array}{ll}
                           \begin{array}{c}
                                {#1} \\ \hline
                                {#2}
                           \end{array} ~  (\mbox{#3})
                        \end{array}$}

\newtheorem{theorem}{Theorem}
\newtheorem{lemma}{Lemma}
\newtheorem{definition}{Definition}
\newtheorem{corollary}{Corollary}
\newtheorem{proposition}{Proposition}

%\newenvironment{proof}{\par\noindent {\em Proof:}}{\hfill $\Box$}

\newcommand{\substate}[1]{\par\noindent {\em #1}.\hspace{0.3cm}}
\newcommand{\remark}{\substate{Remark}}
\newcommand{\example}{\substate{Example}}

\newcommand{\elred}[2]{\diagram #1 \rto^{\mbox{El}} & #2 \enddiagram}
\newcommand{\elnf}[1]{\downarrow^{\mbox{El}} {#1}}
\newcommand{\inred}[2]{\diagram #1 \rto^{\mbox{In}} & #2 \enddiagram}
\newcommand{\innf}[1]{\downarrow^{\mbox{In}} {#1}}

\newtheorem{spec}{Specification}[chapter]
\newcommand{\specification}[2]{\begin{spec}\label{#2} #1.
                                \par\noindent\hrulefill
                                \begin{tgrind}
                                \input{#2}
                                \end{tgrind}
                                \par\noindent\hrulefill
                                \end{spec}}



